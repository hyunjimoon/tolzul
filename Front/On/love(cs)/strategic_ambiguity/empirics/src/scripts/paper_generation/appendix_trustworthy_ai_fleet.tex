\documentclass[11pt]{article}
\usepackage[utf8]{inputenc}
\usepackage{kotex}
\usepackage{booktabs}
\usepackage{array}
\usepackage{geometry}
\usepackage{amsmath}
\usepackage{xcolor}
\usepackage{tcolorbox}
\usepackage{enumitem}
\usepackage{graphicx}

\geometry{margin=1in}

\title{Appendix: Building a Trustworthy AI Fleet\\
\large A Modular Protocol Integrating Product Development, 見利思義, and Bayesian Workflow}
\author{Moon Hyunji}
\date{November 2025}

\begin{document}
\maketitle

\begin{tcolorbox}[colback=gray!10, colframe=gray!50, title=How to Use This Appendix]
This appendix is a \textbf{portable module}. Copy into your LLM context to instantiate a collaborative AI fleet. Trust emerges from \textbf{structure}---explicit roles, quality gates, and verified handoffs.
\end{tcolorbox}

\section{Conceptual Foundations}

This protocol synthesizes three intellectual traditions to enable genuine collaboration among AI agents and human principals.

\subsection{Product Development Process (Ulrich \& Eppinger)}

We adopt the \textbf{Complex System Development Process} for managing parallel subsystems. Our three papers (U, C, N) operate as parallel subsystems with independent design-test cycles before integration.

We employ the \textbf{Heavyweight Project Matrix Organization}. The human Commander (Tongjesa) holds project authority, while functional expertise is distributed across 13 specialized agents.

\subsection{見利思義 (Gyeonri-Saui): Confucian Virtue Framework}

``When you see profit, think of righteousness.'' Decomposed into operational virtues:
\begin{itemize}[leftmargin=2cm]
    \item[\textbf{見}] \textbf{(Observe)}: Ground truth preservation (Commander)
    \item[\textbf{利}] \textbf{(Speed)}: Rapid prototyping (Jeong Agents)
    \item[\textbf{思}] \textbf{(Structure)}: Deep analysis (Na Agents)
    \item[\textbf{義}] \textbf{(Rigor)}: Critical verification (Kim Agents)
\end{itemize}

\section{The 13-Agent Architecture}

The fleet is organized into three functional divisions, mirroring the MIT functional mapping.

\begin{table}[h]
\centering
\begin{tabular}{@{}lllllll@{}}
\toprule
\textbf{Division} & \textbf{Agents} & \textbf{Role} & \textbf{Virtue} & \textbf{Bayesian Role} & \textbf{RP} & \textbf{Color} \\
\midrule
\textbf{Jeong Busa} & J-Intro, J-Theory & Marketing & 利 (Speed) & Prior $\pi(\theta)$ & RP0-1 & \textcolor{green}{Turtle Green} \\
(Concept) & J-Empirics, J-Discuss & (Generator) & & & & \\
\midrule
\textbf{Na Busa} & N-Intro, N-Theory & Manufacturing & 造 (Build) & Posterior $\hat{\pi}(\theta|y)$ & RP2 & \textcolor{orange}{Tiger Orange} \\
(Build) & N-Empirics, N-Discuss & (Builder) & & & & \\
\midrule
\textbf{Kim Busa} & K-Ushape (U) & QA / Verify & 義 (Rigor) & Calibration $T(y,\theta)$ & RP3 & \textcolor{magenta}{Verify Pink} \\
(Verify) & K-Commit (C) & (Verifier) & & & & \\
 & K-News (N) & & & & & \\
\midrule
\textbf{Eo Young-dam} & Obsidian & Memory / Log & 見 (Observe) & Generator $\pi_{joint}$ & RP4 & \textcolor{gray}{Obsidian Gray} \\
(Memory) & (Logs) & (Recorder) & & & & \\
\bottomrule
\end{tabular}
\caption{The Jeolla Fleet Organization}
\end{table}

\section{Workflow Dynamics (The Factory Floor)}

The production line follows a strict sequence, transforming raw "improvisation" into "robust theory":

\begin{enumerate}
    \item \textbf{Input (Eo Young-dam)}: Captures transcripts, raw logs, and "The Uncertainty Engine" drafts.
    \item \textbf{Concept (Jeong Agents)}:
    \begin{itemize}
        \item J-Intro: Story Hook
        \item J-Theory: Model Explanation
        \item J-Empirics: Hypothesis
        \item J-Discuss: Implications
    \end{itemize}
    \item \textbf{Build (Na Agents)}:
    \begin{itemize}
        \item N-Theory: Mathematical Implementation
        \item N-Empirics: Simulation Code \& Figures
    \end{itemize}
    \item \textbf{Verify (Kim Agents)}: Checks for "Cheap Talk" vs "Robustness".
    \item \textbf{Decision (Moon)}: Final Approval or Revision Order.
\end{enumerate}

\section{System Validation (MFS View Evaluation)}

\begin{itemize}
    \item \textbf{Resource Allocation}: The Matrix structure visually separates P1/P2/P3 subsystems, preventing logical cross-contamination.
    \item \textbf{Ownership}: Color coding (Green/Orange/Red) enforces clear decision boundaries and prevents responsibility avoidance.
    \item \textbf{Measurability}: Paragraph counts (dots) quantify the "Remaining Audit Steps" for each Rally Point, enabling precise risk control.
\end{itemize}

\subsection{The Kim Agents (The Red Team)}

The Kim Agents enforce department-specific standards, acting as the "Red Team" (Reviewer \#2).

\begin{itemize}
    \item \textbf{K-Ushape (Entrepreneurship)}: Demands \textit{Innovation, Scaling, Relevance}.
    \item \textbf{K-Commit (Strategy)}: Demands \textit{Advantage, Performance, Mechanism}.
    \item \textbf{K-News (Operations)}: Demands \textit{Efficiency, Rigor, Model}.
\end{itemize}

\section{Rally Point System (Quality Gates)}

Adapted from Tyco's nine-phase process.

\begin{table}[h]
\centering
\begin{tabular}{@{}llll@{}}
\toprule
\textbf{RP} & \textbf{Phase} & \textbf{Owner} & \textbf{Pass Criterion} \\
\midrule
RP0 & Concept Definition & Jeong Agents & Hook + RQ defined \\
RP1 & Preliminary Design & Jeong Agents & Hypothesis + "They Say/I Say" \\
RP2 & Final Design & Na Agents & Code reproducible + Structure \\
RP3 & Product Verification & Kim Agents & \textbf{Risk Score $<$ 20} (Green) \\
RP4 & Process Verification & Commander & Dashboard Review (MFS View) \\
RP5 & Launch Approval & Commander & Final Submission \\
\bottomrule
\end{tabular}
\caption{Rally Point Quality Gates}
\end{table}

\section{The Critique Protocol (Automated Review)}

The \texttt{critique\_spaceship.py} module enforces a rigorous penalty system:

\begin{enumerate}
    \item \textbf{Structure Check}: Missing headers? (+30 Penalty)
    \item \textbf{Vagueness Check}: Too many "weasel words"? (+20 Penalty)
    \item \textbf{Rhetorical Check}: Missing "They Say / I Say" moves? (+15 Penalty)
    \item \textbf{Novelty Check (Posen-Cachon)}: Missing "Surprise" (Null-Breaking)? (+15 Penalty)
    \item \textbf{Department Check}: Missing Kim Agent keywords? (+20 Penalty)
\end{enumerate}

\section{Unified Formula (Bayesian Integration)}

\begin{equation}
k^* = F_D^{-1}\left(\frac{C}{C+F}\right)
\end{equation}

Where P1 (U) determines $D$ (futures), P2 (C) determines $C/F$ (costs), and P3 (N) derives $k^*$ (optimal options).

\end{document}
