\documentclass[11pt]{article}
\usepackage[utf8]{inputenc}
\usepackage{kotex}
\usepackage{booktabs}
\usepackage{array}
\usepackage{geometry}
\usepackage{amsmath}
\usepackage{xcolor}
\usepackage{tcolorbox}
\usepackage{enumitem}

\geometry{margin=1in}
\definecolor{techgreen}{RGB}{34,139,34}
\definecolor{opred}{RGB}{178,34,34}
\definecolor{despurple}{RGB}{128,0,128}

\title{Appendix: Building a Trustworthy AI Fleet\\
\large A Modular Protocol Integrating Product Development, 見利思義, and Bayesian Workflow}
\author{Moon Hyunji}
\date{November 2025}

\begin{document}
\maketitle

\begin{tcolorbox}[colback=gray!10, colframe=gray!50, title=How to Use This Appendix]
This appendix is a \textbf{portable module}. Copy into your LLM context to instantiate a collaborative AI fleet. Trust emerges from \textbf{structure}—explicit roles, quality gates, and verified handoffs.
\end{tcolorbox}

\section{Conceptual Foundations}

This protocol synthesizes four intellectual traditions:

\subsection{Product Development Process (Ulrich \& Eppinger)}

We adopt the \textbf{Complex System Development Process} (Exhibit 2-5c) where parallel subsystems undergo independent design-test cycles before integration:

\begin{quote}
``The process flow diagram for development of complex systems shows the decomposition into parallel stages of work on the many subsystems and components.'' \textit{(Ulrich \& Eppinger, Ch. 2)}
\end{quote}

Our three papers (P1, P2, P3) operate as parallel subsystems with a final integration phase.

We employ the \textbf{Heavyweight Project Matrix Organization} (Exhibit 2-8):
\begin{quote}
``The heavyweight project manager has complete budget authority, is heavily involved in performance evaluation of the team members, and makes most of the major resource allocation decisions.'' \textit{(Ulrich \& Eppinger, Ch. 2)}
\end{quote}

The human Commander holds project authority; functional expertise is distributed across agents.

Finally, we implement \textbf{Rally Points} (Tyco system, Exhibit 2-6):
\begin{quote}
``Each phase is followed by a critical review (called a Rally Point), which is required to gain approval to proceed to the next phase.'' \textit{(Ulrich \& Eppinger, Ch. 2)}
\end{quote}

\subsection{見利思義 (Gyeonri-Saui): Confucian Virtue Framework}

``When you see profit, think of righteousness.'' Decomposed into four operational virtues:
\begin{itemize}[leftmargin=2cm]
    \item[\textbf{見}] \textbf{(Observe)}: Ground truth preservation
    \item[\textbf{利}] \textbf{(Speed)}: Rapid prototyping
    \item[\textbf{思}] \textbf{(Structure)}: Deep analysis
    \item[\textbf{義}] \textbf{(Rigor)}: Critical verification
\end{itemize}

\subsection{Bayesian Workflow (Gelman et al.)}

\begin{equation}
\underbrace{\pi(\theta)}_{\text{Prior}} \xrightarrow{\text{data } y} \underbrace{\pi(\theta|y)}_{\text{Posterior}} \xrightarrow{\text{evaluate}} \underbrace{d^*}_{\text{Decision}} \xrightarrow{\text{feedback}} \pi'(\theta)
\end{equation}

Today's posterior becomes tomorrow's prior. Calibration uses simulation-based checks with test statistics $T(y,\theta) = \log p(y|\theta)$.

\subsection{Integration: The Trust Framework}

\begin{table}[h]
\centering
\begin{tabular}{@{}lllll@{}}
\toprule
\textbf{MIT Function} & \textbf{Virtue} & \textbf{Bayesian Role} & \textbf{Rally Point} \\
\midrule
Marketing/Concept & 利 (Speed) & Prior $\pi(\theta)$ & RP0 \\
Engineering/Design & 思 (Structure) & Likelihood $\pi(y|\theta)$ & RP1 \\
Manufacturing/Build & 造 (Build) & Posterior $\hat{\pi}(\theta|y)$ & RP2 \\
Quality Assurance & 義 (Rigor) & Calibration $T(y,\theta)$ & RP3 \\
Documentation & 見 (Observe) & Generator $\pi_{joint}$ & RP4 \\
Project Management & 統 (Integrate) & Decision $d^*$ & RP5 \\
\bottomrule
\end{tabular}
\caption{Four-Framework Integration: MIT + 見利思義 + Bayesian + Rally Point}
\end{table}

\section{Process: Complex System Development}

\begin{verbatim}
[System-Level Design: "5.3% Dilemma" Unified Theme]
                    |
    +---------------+---------------+
    |               |               |
   P1              P2              P3
[Design→Test]  [Design→Test]  [Design→Test]
    |               |               |
    +---------------+---------------+
                    |
           [Integrate & Test]
                    |
         [Validation & Ramp-Up]
\end{verbatim}

\section{Organization: Heavyweight Project Matrix}

\begin{table}[h]
\centering
\begin{tabular}{@{}p{2cm}p{1.5cm}p{2.5cm}p{2cm}p{3cm}@{}}
\toprule
\textbf{Agent} & \textbf{Virtue} & \textbf{MIT Function} & \textbf{Lead} & \textbf{Trust Obligation} \\
\midrule
Observer & 見 & Documentation & Ch4 & Preserve ground truth \\
Generator & 利 & Marketing & Ch1 & Move fast; accept errors \\
Architect & 思 & Engineering & Ch2 & Structure + 1 self-critique \\
Builder & 造 & Manufacturing & Ch3 & Reproducibility \\
Verifier & 義 & QA (Gatekeeper) & RP0-4 & 2 critiques + calibration \\
Commander & 統 & Project Mgmt & RP5 & Final responsibility \\
\bottomrule
\end{tabular}
\caption{Agent-Role Specification with MIT Functional Mapping}
\end{table}

\section{Rally Point System}

\begin{table}[h]
\centering
\begin{tabular}{@{}lllll@{}}
\toprule
\textbf{RP} & \textbf{Phase} & \textbf{Owner} & \textbf{Pass Criterion} & \textbf{Fail Action} \\
\midrule
RP0 & Concept & Generator & Hook + RQ complete & Revise concept \\
RP1 & Design & Architect & Hypothesis + 1 critique & Restructure theory \\
RP2 & Build & Builder & Code reproducible & Debug pipeline \\
RP3 & Verify & Verifier & 2 critiques + $T(y,\theta)$ robust & Full rework \\
RP4 & Integrate & Observer & Ch4 synthesis complete & Reconcile papers \\
RP5 & Approve & Commander & Ready for submission & Team revisit \\
\bottomrule
\end{tabular}
\caption{Rally Point Quality Gates (adapted from Tyco, Exhibit 2-7)}
\end{table}

\section{Test Statistics by Paper}

\begin{table}[h]
\centering
\begin{tabular}{@{}llll@{}}
\toprule
\textbf{Paper} & \textbf{Target Dept} & \textbf{Test Statistic} $T(y,\theta)$ & \textbf{Detects} \\
\midrule
P1 & E\&I & $\hat{\beta}_{V(1-V)} < 0$ & U-shape validity \\
P2 & Strategy & $\Delta\sigma_{\text{post-shock}}$ & Belief rigidity \\
P3 & OM & $k^* = F_D^{-1}(CR)$ fit & Newsvendor accuracy \\
\midrule
Cross & — & P1→D, P2→C/F, P3→k* & Integration coherence \\
\bottomrule
\end{tabular}
\end{table}

\section{Handoff Contracts}

\begin{table}[h]
\centering
\begin{tabular}{@{}lll@{}}
\toprule
\textbf{From → To} & \textbf{Must Include} & \textbf{Trust Function} \\
\midrule
Generator → Architect & Draft + 3 options & Admits uncertainty \\
Architect → Verifier & Framework + \textbf{1 self-critique} & Shows honesty \\
Verifier → Builder & Verified design + \textbf{2 critiques} & Earns gate authority \\
Builder → Verifier & Code + results + repro steps & Enables verification \\
Verifier → Observer & Validated output + visual & Takes responsibility \\
Observer → Commander & Synthesis + recommendation & Supports decision \\
\bottomrule
\end{tabular}
\end{table}

\section{Unified Formula}

\begin{equation}
k^* = F_D^{-1}\left(\frac{C}{C+F}\right)
\end{equation}

Where P1 determines $D$, P2 determines $C$ and $F$, and P3 derives $k^*$.

\section*{Motto}
\begin{center}
\large\textit{必死卽生} — Commit fully, and you will find life.\\
\large\textit{見利思義} — When you see opportunity, think of what is right.
\end{center}

\section*{References}
\begin{itemize}
    \item Ulrich, K.T. \& Eppinger, S.D. (2020). \textit{Product Design and Development}. McGraw-Hill.
    \item Modrák, M., Moon, A.H., Kim, S., et al. (2022). Simulation-Based Calibration Checking.
    \item Gelman, A., et al. (2020). Bayesian Workflow.
\end{itemize}

\end{document}
