% Auto-generated Results section
% Module #23-27

\section{Results}

\subsection{H1: Early Funding (OLS)}
% Paragraph 23: H1 results
For H1, the coefficient on vagueness is \textcolor{blue}{statistically significant and negative}: $\hat\beta_1 \approx -8.51e-07$ ($p=0.000$). This supports the information cost hypothesis: vague promises reduce early funding by preventing investors from updating beliefs through verifiable milestones.

\begin{threeparttable}[htb!]
\caption{Model A: Early Funding (OLS)}
\label{tab:T1_ModelA_EarlyFunding}
\begin{tabular}{l c}
\toprule
 & Early funding (M USD) \\
\midrule
Constant & 4.76e-06*** (5.47e-07) \\
Vagueness (z) & -8.51e-07*** (2.32e-07) \\
Employees (log, z) & 2.83e-06*** (2.06e-07) \\
\midrule
Sector FE  & Yes \\
N          & 43,126 \\
$R^2$      & 0.010 \\
\bottomrule
\end{tabular}
\begin{tablenotes}[flushleft]
\item Notes: Coefficients with robust standard errors in parentheses.
Significance: $^{***}p<0.001$, $^{**}p<0.01$, $^{*}p<0.05$, $^{\dagger}p<0.10$.
\end{tablenotes}
\end{threeparttable}


\subsection{H2: Growth to Series B+ (Logit)}
% Paragraph 24: H2 main effect
For H2, we find a statistically \textcolor{blue}{significant negative main effect of vagueness on growth}: $\hat\alpha_1 \approx -0.0372$ ($p=0.0000$).
The interaction with hardware is negative and statistically significant at the 10\% level: $\hat\alpha_3 \approx -0.0300$ ($p=0.046$). This suggests that exercisability moderates the effect of vagueness on later success.

\begin{threeparttable}[htb!]
\caption{Model B: Later Success (Logit)}
\label{tab:T2_ModelB_LaterSuccess}
\begin{tabular}{l c}
\toprule
 & Series B+ (log-odds) \\
\midrule
Constant & -4.328*** (0.012) \\
Z Vagueness & -0.037*** (0.007) \\
Is Hardware & 0.448*** (0.014) \\
Vagueness × Hardware & -0.030* (0.015) \\
\midrule
Cohort FE & Yes \\
N & 1,052,481 \\
Pseudo-$R^2$ & 0.146 \\
\bottomrule
\end{tabular}
\begin{tablenotes}[flushleft]
\item Notes: Logit coefficients (log-odds) with robust standard errors in parentheses.
Significance: $^{***}p<0.001$, $^{**}p<0.01$, $^{*}p<0.05$, $^{\dagger}p<0.10$.
\end{tablenotes}
\end{threeparttable}


% Figures
\begin{figure}[p]
    \centering
    \includegraphics[width=0.9\linewidth]{paper/figures/fig2_early_funding.pdf}
    \caption{Early Funding vs. Vagueness (H1)}
    \label{fig:Fig2_EVF}
\end{figure}

\begin{figure}[p]
    \centering
    \includegraphics[width=0.9\linewidth]{paper/figures/fig3_later_success.pdf}
    \caption{Later Success vs. Vagueness by Exercisability (H2)}
    \label{fig:Fig3_LVF}
\end{figure}
