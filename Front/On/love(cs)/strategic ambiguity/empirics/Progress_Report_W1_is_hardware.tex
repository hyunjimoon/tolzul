% Options for packages loaded elsewhere
\PassOptionsToPackage{unicode}{hyperref}
\PassOptionsToPackage{hyphens}{url}
\PassOptionsToPackage{dvipsnames,svgnames,x11names}{xcolor}
%
\documentclass[
  10pt,
  letterpaper,
  DIV=11,
  numbers=noendperiod]{scrartcl}

\usepackage{amsmath,amssymb}
\usepackage{lmodern}
\usepackage{iftex}
\ifPDFTeX
  \usepackage[T1]{fontenc}
  \usepackage[utf8]{inputenc}
  \usepackage{textcomp} % provide euro and other symbols
\else % if luatex or xetex
  \usepackage{unicode-math}
  \defaultfontfeatures{Scale=MatchLowercase}
  \defaultfontfeatures[\rmfamily]{Ligatures=TeX,Scale=1}
\fi
% Use upquote if available, for straight quotes in verbatim environments
\IfFileExists{upquote.sty}{\usepackage{upquote}}{}
\IfFileExists{microtype.sty}{% use microtype if available
  \usepackage[]{microtype}
  \UseMicrotypeSet[protrusion]{basicmath} % disable protrusion for tt fonts
}{}
\makeatletter
\@ifundefined{KOMAClassName}{% if non-KOMA class
  \IfFileExists{parskip.sty}{%
    \usepackage{parskip}
  }{% else
    \setlength{\parindent}{0pt}
    \setlength{\parskip}{6pt plus 2pt minus 1pt}}
}{% if KOMA class
  \KOMAoptions{parskip=half}}
\makeatother
\usepackage{xcolor}
\usepackage[margin=0.75in]{geometry}
\setlength{\emergencystretch}{3em} % prevent overfull lines
\setcounter{secnumdepth}{5}
% Make \paragraph and \subparagraph free-standing
\ifx\paragraph\undefined\else
  \let\oldparagraph\paragraph
  \renewcommand{\paragraph}[1]{\oldparagraph{#1}\mbox{}}
\fi
\ifx\subparagraph\undefined\else
  \let\oldsubparagraph\subparagraph
  \renewcommand{\subparagraph}[1]{\oldsubparagraph{#1}\mbox{}}
\fi


\providecommand{\tightlist}{%
  \setlength{\itemsep}{0pt}\setlength{\parskip}{0pt}}\usepackage{longtable,booktabs,array}
\usepackage{calc} % for calculating minipage widths
% Correct order of tables after \paragraph or \subparagraph
\usepackage{etoolbox}
\makeatletter
\patchcmd\longtable{\par}{\if@noskipsec\mbox{}\fi\par}{}{}
\makeatother
% Allow footnotes in longtable head/foot
\IfFileExists{footnotehyper.sty}{\usepackage{footnotehyper}}{\usepackage{footnote}}
\makesavenoteenv{longtable}
\usepackage{graphicx}
\makeatletter
\def\maxwidth{\ifdim\Gin@nat@width>\linewidth\linewidth\else\Gin@nat@width\fi}
\def\maxheight{\ifdim\Gin@nat@height>\textheight\textheight\else\Gin@nat@height\fi}
\makeatother
% Scale images if necessary, so that they will not overflow the page
% margins by default, and it is still possible to overwrite the defaults
% using explicit options in \includegraphics[width, height, ...]{}
\setkeys{Gin}{width=\maxwidth,height=\maxheight,keepaspectratio}
% Set default figure placement to htbp
\makeatletter
\def\fps@figure{htbp}
\makeatother

\usepackage{graphicx}
\usepackage{grffile}
\usepackage{adjustbox}
\usepackage{booktabs}
\usepackage{longtable}
\usepackage{array}
\usepackage{multirow}
\usepackage{wrapfig}
\usepackage{float}
\usepackage{colortbl}
\usepackage{pdflscape}
\usepackage{tabu}
\usepackage{threeparttable}
\usepackage{threeparttablex}
\usepackage[normalem]{ulem}
\usepackage{makecell}
\usepackage{xcolor}
\KOMAoption{captions}{tableheading}
\makeatletter
\makeatother
\makeatletter
\makeatother
\makeatletter
\@ifpackageloaded{caption}{}{\usepackage{caption}}
\AtBeginDocument{%
\ifdefined\contentsname
  \renewcommand*\contentsname{Table of contents}
\else
  \newcommand\contentsname{Table of contents}
\fi
\ifdefined\listfigurename
  \renewcommand*\listfigurename{List of Figures}
\else
  \newcommand\listfigurename{List of Figures}
\fi
\ifdefined\listtablename
  \renewcommand*\listtablename{List of Tables}
\else
  \newcommand\listtablename{List of Tables}
\fi
\ifdefined\figurename
  \renewcommand*\figurename{Figure}
\else
  \newcommand\figurename{Figure}
\fi
\ifdefined\tablename
  \renewcommand*\tablename{Table}
\else
  \newcommand\tablename{Table}
\fi
}
\@ifpackageloaded{float}{}{\usepackage{float}}
\floatstyle{ruled}
\@ifundefined{c@chapter}{\newfloat{codelisting}{h}{lop}}{\newfloat{codelisting}{h}{lop}[chapter]}
\floatname{codelisting}{Listing}
\newcommand*\listoflistings{\listof{codelisting}{List of Listings}}
\makeatother
\makeatletter
\@ifpackageloaded{caption}{}{\usepackage{caption}}
\@ifpackageloaded{subcaption}{}{\usepackage{subcaption}}
\makeatother
\makeatletter
\@ifpackageloaded{tcolorbox}{}{\usepackage[many]{tcolorbox}}
\makeatother
\makeatletter
\@ifundefined{shadecolor}{\definecolor{shadecolor}{rgb}{.97, .97, .97}}
\makeatother
\makeatletter
\makeatother
\ifLuaTeX
  \usepackage{selnolig}  % disable illegal ligatures
\fi
\IfFileExists{bookmark.sty}{\usepackage{bookmark}}{\usepackage{hyperref}}
\IfFileExists{xurl.sty}{\usepackage{xurl}}{} % add URL line breaks if available
\urlstyle{same} % disable monospaced font for URLs
\hypersetup{
  pdftitle={Progress Report: Strategic Ambiguity in Venture Capital},
  pdfauthor={Research Team},
  colorlinks=true,
  linkcolor={blue},
  filecolor={Maroon},
  citecolor={Blue},
  urlcolor={Blue},
  pdfcreator={LaTeX via pandoc}}

\title{Progress Report: Strategic Ambiguity in Venture Capital}
\usepackage{etoolbox}
\makeatletter
\providecommand{\subtitle}[1]{% add subtitle to \maketitle
  \apptocmd{\@title}{\par {\large #1 \par}}{}{}
}
\makeatother
\subtitle{W1 Hypothesis Testing with Architecture (Integration Cost)
Moderator}
\author{Research Team}
\date{10/29/25}

\begin{document}
\maketitle
\ifdefined\Shaded\renewenvironment{Shaded}{\begin{tcolorbox}[interior hidden, frame hidden, boxrule=0pt, enhanced, sharp corners, borderline west={3pt}{0pt}{shadecolor}, breakable]}{\end{tcolorbox}}\fi

\renewcommand*\contentsname{Table of contents}
{
\hypersetup{linkcolor=}
\setcounter{tocdepth}{3}
\tableofcontents
}
\textbf{To:} Professors Charlie Fine and Scott Stern \textbf{From:}
Research Team \textbf{Date:}
\texttt{\{python\}\ from\ datetime\ import\ datetime;\ print(datetime.now().strftime("\%B\ \%d,\ \%Y"))}
\textbf{Subject:} Dependent Variable Validation, Model Specifications,
and Preliminary Results (Architecture Moderator)

\begin{center}\rule{0.5\linewidth}{0.5pt}\end{center}

\hypertarget{executive-summary}{%
\section{Executive Summary}\label{executive-summary}}

This report documents our progress on testing the \textbf{Strategic
Ambiguity} hypothesis framework with \textbf{integration cost}
(architecture) as the primary moderator. Following Professor Stern's
methodological guidance, we have implemented the \textbf{Series A(t₀) →
Series B+(t₁)} framework for defining growth trajectories and validated
our dependent variable construction against theoretical requirements.

\hypertarget{key-accomplishments}{%
\subsection{Key Accomplishments}\label{key-accomplishments}}

\begin{enumerate}
\def\labelenumi{\arabic{enumi}.}
\item
  \textbf{DV Validation}: Confirmed that our H2/H4 dependent variable
  (Series B+ progression) correctly implements the at-risk cohort
  framework (companies at Series A at baseline, excluding prior B+
  rounds).
\item
  \textbf{Methodological Refinement}: Implemented Series A filtering
  using PitchBook's ``Early Stage VC'' label to isolate true
  venture-backed companies and exclude angel/seed rounds from H1/H3
  analysis.
\item
  \textbf{Integration Cost Moderator}: Tested \texttt{is\_hardware} as a
  moderator capturing differences between integrated (hardware/biotech)
  and modular (software) architectures.
\item
  \textbf{One-Touch Execution System}: Created automated pipeline for
  reproducibility and rapid iteration.
\end{enumerate}

\hypertarget{results-summary}{%
\subsection{Results Summary}\label{results-summary}}

\hypertarget{results-summary}{}
\begin{verbatim}

**H1 (Early Funding)**: Vagueness coefficient = -0.0000 (p = 0.2078)

**H2 (Growth × Architecture)**: Interaction = 0.0925 (p = 0.0622)

**H3/H4 (Architecture Moderator)**: Conceptually defined below (Section 5)
\end{verbatim}

\hypertarget{methodological-limitations-identified}{%
\subsection{Methodological Limitations
Identified}\label{methodological-limitations-identified}}

\hypertarget{limitations}{}
\begin{verbatim}

**17-month follow-up constraint**: Our observation window (December 2021 – May 2023) captures only 17 months of post-Series A progression, well below the recommended 46-month window. This introduces right censoring (≈50-60% false negatives) and restricts our findings to "rapid progressors."

**Cohort heterogeneity**: The at-risk cohort for H2/H4 does not filter by founding year, introducing potential vintage effects that may confound growth trajectories.

These limitations are addressed in Section 4 with proposed robustness checks and reframing strategies.
\end{verbatim}

\begin{center}\rule{0.5\linewidth}{0.5pt}\end{center}

\hypertarget{introduction-hypothesis-framework}{%
\section{Introduction \& Hypothesis
Framework}\label{introduction-hypothesis-framework}}

\hypertarget{theoretical-motivation}{%
\subsection{Theoretical Motivation}\label{theoretical-motivation}}

Strategic ambiguity---the deliberate use of vague or broad category
labels---has been theorized as a \textbf{double-edged sword} in
entrepreneurial finance. On one hand, category spanning and ambiguous
positioning can help startups gain initial attention and resources by
appealing to diverse stakeholder groups (Granqvist et al., 2013; Navis
\& Glynn, 2010; Zuckerman, 1999). On the other hand, the same ambiguity
may hinder later-stage scaling by introducing coordination costs,
identity confusion, and valuation uncertainty (Negro \& Leung, 2013; Wry
et al., 2014).

Recent empirical work suggests that vagueness in startup positioning can
serve as a \textbf{strategic signal} rather than mere noise. El-Zayaty
et al.~(2025) find that linguistic ambiguity in pitch decks correlates
with early funding success, while Pan et al.~(2018) demonstrate that
categorical ambiguity enables startups to navigate crowded market
spaces. However, these benefits may not persist across growth stages.

\hypertarget{the-reversal-hypothesis}{%
\subsection{The Reversal Hypothesis}\label{the-reversal-hypothesis}}

We propose a \textbf{stage-contingent reversal pattern} in how vagueness
affects startup success:

\begin{itemize}
\item
  \textbf{H1 (Early Funding, Series A)}: Higher vagueness in startup
  positioning \textbf{positively} predicts Series A funding amounts.
  \emph{Mechanism}: Ambiguity attracts diverse early-stage investors and
  signals flexibility/optionality (Huang et al., 2014; Loughran \&
  McDonald, 2016).
\item
  \textbf{H2 (Growth, Series B+ Progression)}: Higher vagueness in
  startup positioning \textbf{negatively} predicts the likelihood of
  progressing from Series A to Series B+. \emph{Mechanism}: As startups
  scale, investors demand clarity on business model, market fit, and
  competitive positioning. Persistent ambiguity becomes a liability
  (Navis et al., 2023).
\end{itemize}

This reversal pattern aligns with \textbf{dual legitimacy strategies} in
organizational theory: early-stage ventures benefit from ``categorical
flexibility'' to secure resources, but must converge to ``categorical
clarity'' to achieve legitimacy with later-stage stakeholders (Wry et
al., 2014; Zuckerman, 1999).

\hypertarget{boundary-conditions-integration-cost-architecture}{%
\subsection{Boundary Conditions: Integration Cost
(Architecture)}\label{boundary-conditions-integration-cost-architecture}}

We test whether \textbf{integration cost} (captured by hardware
vs.~software architecture) moderates these relationships:

\hypertarget{theoretical-framework}{%
\subsubsection{Theoretical Framework}\label{theoretical-framework}}

\textbf{Integration Cost Hypothesis}: Hardware and biotech startups face
higher integration costs than software startups due to:

\begin{enumerate}
\def\labelenumi{\arabic{enumi}.}
\tightlist
\item
  \textbf{Physical component dependencies}: Manufacturing, supply chain,
  inventory management
\item
  \textbf{Longer development cycles}: Prototyping, testing, regulatory
  approval (especially biotech)
\item
  \textbf{Capital intensity}: Equipment, facilities, specialized labor
\item
  \textbf{Lower pivot flexibility}: Sunk costs in physical
  infrastructure
\end{enumerate}

These constraints limit \textbf{strategic flexibility}, making vague
positioning \textbf{more costly} for hardware startups. Conversely,
software startups (modular architectures) can pivot quickly, making
vague positioning a viable exploration strategy.

\hypertarget{hypotheses}{%
\subsubsection{Hypotheses}\label{hypotheses}}

\begin{itemize}
\item
  \textbf{H3 (Early Funding × Architecture)}: Integration cost
  attenuates the positive effect of vagueness on Series A funding.
  \emph{Logic}: Hardware startups cannot afford vague positioning even
  at early stages due to capital intensity and longer timelines.
  Investors demand clarity on technical feasibility and market
  application.
\item
  \textbf{H4 (Growth × Architecture)}: Integration cost amplifies the
  negative effect of vagueness on Series B+ progression. \emph{Logic}:
  By Series B+, hardware startups with unclear positioning face
  compounding risks: product-market fit uncertainty plus technical
  execution risk. Software startups can tolerate more ambiguity during
  scaling.
\end{itemize}

\textbf{Expected Pattern}: Negative interaction (vagueness helps more in
software/modular architectures).

These hypotheses draw on \textbf{modularity theory} (Baldwin \& Clark,
2000) and \textbf{architectural innovation} literature (Henderson \&
Clark, 1990), applied to entrepreneurial finance contexts.

\begin{center}\rule{0.5\linewidth}{0.5pt}\end{center}

\hypertarget{data-measurement}{%
\section{Data \& Measurement}\label{data-measurement}}

\hypertarget{data-source}{%
\subsection{Data Source}\label{data-source}}

We use \textbf{PitchBook proprietary data} covering venture-backed
companies in the United States. Our analysis leverages four temporal
snapshots to construct longitudinal outcomes and control for as-of
biases:

\hypertarget{data-snapshots}{}
\begin{verbatim}
| Snapshot          | Date             | Purpose                                      |
|:------------------|:-----------------|:---------------------------------------------|
| Baseline (t₀)     | December 1, 2021 | Extract predictors and define at-risk cohort |
| Mid-point 1 (tₘ₁) | January 1, 2022  | Track event ordering                         |
| Mid-point 2 (tₘ₂) | May 1, 2022      | Track event ordering                         |
| Endpoint (t₁)     | May 1, 2023      | Observe final outcomes                       |
\end{verbatim}

\textbf{Observation window}: 17 months post-baseline (December 2021 --
May 2023).

\hypertarget{key-variables}{%
\subsection{Key Variables}\label{key-variables}}

\hypertarget{independent-variable-vagueness}{%
\subsubsection{Independent Variable:
Vagueness}\label{independent-variable-vagueness}}

\begin{itemize}
\tightlist
\item
  \textbf{Operationalization}: Composite measure based on categorical
  breadth, keyword diversity, and positioning clarity in company
  descriptions (normalized as \texttt{z\_vagueness}).
\item
  \textbf{Theoretical grounding}: Loughran \& McDonald (2016) document
  that linguistic ambiguity affects investor perceptions in financial
  disclosures; we extend this to startup positioning.
\end{itemize}

\hypertarget{dependent-variables}{%
\subsubsection{Dependent Variables}\label{dependent-variables}}

\hypertarget{h1h3-early-funding-series-a-amount}{%
\paragraph{H1/H3: Early Funding (Series A
Amount)}\label{h1h3-early-funding-series-a-amount}}

\begin{itemize}
\tightlist
\item
  \textbf{Variable}: \texttt{early\_funding\_musd} (first financing size
  in millions USD)
\item
  \textbf{Filter}: \textbf{PitchBook ``Early Stage VC''} label only
  (excludes Seed, Angel, Later Stage VC)
\item
  \textbf{Rationale}: Isolates true Series A venture rounds.
\item
  \textbf{Analysis}: OLS regression (continuous outcome).
\end{itemize}

\hypertarget{h2h4-growth-series-b-progression}{%
\paragraph{H2/H4: Growth (Series B+
Progression)}\label{h2h4-growth-series-b-progression}}

\begin{itemize}
\tightlist
\item
  \textbf{Variable}: \texttt{growth} (binary: 1 = progressed to Series
  B+, 0 = remained at A/failed)
\item
  \textbf{Framework}: Scott Stern's \textbf{Series A(t₀) → Series
  B+(t₁)} progression model
\item
  \textbf{At-risk cohort}: Companies at Series A stage (VC-backed) at
  baseline with no prior B+ funding
\item
  \textbf{Success event}: Transition to Series B+ financing observed
  after baseline
\item
  \textbf{Censoring}: M\&A exits treated as competing risk (censored,
  not failures)
\item
  \textbf{Analysis}: Logistic regression (binary outcome).
\end{itemize}

\hypertarget{moderator-integration-cost-is_hardware}{%
\subsubsection{Moderator: Integration Cost
(is\_hardware)}\label{moderator-integration-cost-is_hardware}}

\hypertarget{moderator-definition}{}
\begin{verbatim}

**Operationalization**: Binary indicator (`is_hardware = 1` for hardware/biotech, `0` for software)

**Classification criteria**:
- Hardware (1): Companies with physical products, manufacturing operations, or biotech/pharma focus
- Software (0): Pure software, SaaS, platforms, digital services

**Sample distribution**:
- Hardware/integrated: 11.0% (N = 4,702)
- Software/modular: 89.0% (N = 37,983)

**Theoretical grounding**: Modularity theory (Baldwin & Clark, 2000) distinguishes integrated vs. modular architectures based on component interdependencies and interface standards. Hardware startups exhibit higher integration costs due to physical constraints.
\end{verbatim}

\hypertarget{controls}{%
\subsubsection{Controls}\label{controls}}

\hypertarget{controls-table}{}
\begin{verbatim}
| Control Variable     | Operationalization                                             | Purpose                                                |
|:---------------------|:---------------------------------------------------------------|:-------------------------------------------------------|
| Firm size            | z_employees_log (log-transformed employee count, standardized) | Control for firm scale and resources                   |
| Founding cohort      | Categorical fixed effects for founding year bins               | Control for vintage effects and macro conditions       |
| Sector fixed effects | Industry classification (H1/H3 only)                           | Account for sector-specific norms and funding patterns |
\end{verbatim}

\hypertarget{as-of-date-capping}{%
\subsection{As-of Date Capping}\label{as-of-date-capping}}

A critical data quality issue: PitchBook snapshots contain
\textbf{future-dated events} due to retroactive data entry. To prevent
\textbf{data leakage}, we implemented \textbf{as-of date capping}:

\begin{itemize}
\tightlist
\item
  For each snapshot date, we cap \texttt{LastFinancingDate} to the
  snapshot date
\item
  We retain \texttt{LastFinancingDealType} (current financing stage)
  even when dates exceed snapshot cutoffs
\end{itemize}

This methodological correction is documented in \texttt{features.py}
(lines 242-282) and prevents upward bias in outcome measurement.

\begin{center}\rule{0.5\linewidth}{0.5pt}\end{center}

\hypertarget{dependent-variable-validation-refinement}{%
\section{Dependent Variable Validation \&
Refinement}\label{dependent-variable-validation-refinement}}

\hypertarget{scott-sterns-guidance-series-atux2080-series-btux2081-framework}{%
\subsection{Scott Stern's Guidance: Series A(t₀) → Series B+(t₁)
Framework}\label{scott-sterns-guidance-series-atux2080-series-btux2081-framework}}

\textbf{Context}: Early iterations of our analysis encountered
\textbf{singular matrix errors} during logistic regression. Professor
Stern advised us to adopt a \textbf{success-oriented progression
framework} rather than survival analysis, focusing on companies that
successfully transition from one financing stage to the next.

\hypertarget{framework-requirements}{%
\subsubsection{Framework Requirements}\label{framework-requirements}}

\hypertarget{framework-requirements}{}
\begin{verbatim}
| Requirement                      | Implementation                                                                  | Status        |
|:---------------------------------|:--------------------------------------------------------------------------------|:--------------|
| 1. At-risk cohort definition     | Identify companies at Series A stage at baseline (t₀)                           | ✓ Implemented |
| 2. No backward contamination     | Exclude companies that already reached Series B+ before baseline                | ✓ Implemented |
| 3. Success event                 | Code Y=1 for companies that transition from Series A → Series B+ after baseline | ✓ Implemented |
| 4. Censoring for competing risks | M&A exits censored (coded as missing, not failures)                             | ✓ Implemented |
| 5. Temporal ordering             | Use multiple snapshots to determine "first seen" dates                          | ✓ Implemented |
\end{verbatim}

\textbf{Why this framework resolves singular matrix errors}: By focusing
on a \textbf{homogeneous at-risk cohort} (all starting at Series A) and
tracking a well-defined transition event (progression to B+), we achieve
sufficient outcome variance.

\hypertarget{h2h4-dv-validation}{%
\subsection{H2/H4 DV Validation}\label{h2h4-dv-validation}}

\textbf{Validation Question}: Does the
\texttt{create\_survival\_seriesb\_progression()} function correctly
implement the at-risk cohort framework?

\hypertarget{findings}{%
\subsubsection{Findings}\label{findings}}

\hypertarget{dv-validation}{}
\begin{verbatim}
| Component                     | Status    | Implementation                                          |
|:------------------------------|:----------|:--------------------------------------------------------|
| Series A stage identification | ✓ Correct | Uses A_STAGE_PAT regex to match "Early Stage VC" label  |
| No backward contamination     | ✓ Correct | Excludes companies with prior B+ via no_prior_b filter  |
| Success event timing          | ✓ Correct | Requires b_idx >= 1 (Series B+ appeared AFTER baseline) |
| M&A censoring                 | ✓ Correct | M&A exits assigned Y_primary = NaN (censored)           |
\end{verbatim}

\hypertarget{methodological-limitation-cohort-heterogeneity}{%
\subsubsection{⚠️ Methodological Limitation: Cohort
Heterogeneity}\label{methodological-limitation-cohort-heterogeneity}}

The at-risk cohort \textbf{does not filter by founding year}. A valid
at-risk cohort should be homogeneous by vintage to avoid confounding
growth trajectories with macroeconomic conditions or cohort effects.

\textbf{Next iteration requirement}: Add a \texttt{founding\_year} or
\texttt{founding\_cohort} filter to ensure comparable maturity at
baseline.

\hypertarget{month-follow-up-period-implications-robustness-checks}{%
\subsection{17-Month Follow-Up Period: Implications \& Robustness
Checks}\label{month-follow-up-period-implications-robustness-checks}}

\textbf{Data constraint}: Our follow-up period spans only 17 months,
substantially shorter than the recommended 46-month window.

\hypertarget{current-base-rate}{%
\subsubsection{Current Base Rate}\label{current-base-rate}}

\hypertarget{base-rate}{}
\begin{verbatim}

**Base rate**: 16.9% of at-risk companies progress to Series B+ within 17 months

- Successful progressions (Y=1): 7,200
- No progression (Y=0): 35,485
- Total at-risk: 42,685

**Reframing**: We focus on **rapid Series B+ progression (≤17 months)** rather than eventual success. This is a theoretically meaningful outcome signaling strong product-market fit.
\end{verbatim}

\begin{center}\rule{0.5\linewidth}{0.5pt}\end{center}

\hypertarget{model-specifications-results}{%
\section{Model Specifications \&
Results}\label{model-specifications-results}}

\hypertarget{h1-early-funding-ols}{%
\subsection{H1: Early Funding (OLS)}\label{h1-early-funding-ols}}

\hypertarget{model-specification}{%
\subsubsection{Model Specification}\label{model-specification}}

\textbf{Research Question}: Does vagueness predict Series A funding
amounts?

\[
\text{early\_funding\_musd} \sim \text{z\_vagueness} + \text{z\_employees\_log} + C(\text{founding\_cohort}) + C(\text{sector\_fe})
\]

\textbf{Estimation}: Ordinary Least Squares (OLS)

\hypertarget{results}{%
\subsubsection{Results}\label{results}}

\hypertarget{h1-results}{}
\begin{verbatim}
#### Coefficient Table

| variable                            |   coefficient |   std_err |    stat |   p_value |   ci_lower |   ci_upper |
|:------------------------------------|--------------:|----------:|--------:|----------:|-----------:|-----------:|
| Intercept                           |        0.0000 |    0.0000 |  3.2206 |    0.0013 |     0.0000 |     0.0000 |
| C(sector_fe)[T.Biotech/Healthcare]  |        0.0000 |    0.0000 |  3.5782 |    0.0003 |     0.0000 |     0.0000 |
| C(sector_fe)[T.Consumer Software]   |       -0.0000 |    0.0000 | -1.7887 |    0.0737 |    -0.0000 |     0.0000 |
| C(sector_fe)[T.Data/Analytics]      |       -0.0000 |    0.0000 | -0.7791 |    0.4359 |    -0.0000 |     0.0000 |
| C(sector_fe)[T.Enterprise Software] |       -0.0000 |    0.0000 | -1.1770 |    0.2392 |    -0.0000 |     0.0000 |
| C(sector_fe)[T.FinTech]             |        0.0000 |    0.0000 |  0.4079 |    0.6834 |    -0.0000 |     0.0000 |
| C(sector_fe)[T.Hardware/Robotics]   |       -0.0000 |    0.0000 | -1.1846 |    0.2362 |    -0.0000 |     0.0000 |
| C(sector_fe)[T.Other]               |       -0.0000 |    0.0000 | -1.0067 |    0.3141 |    -0.0000 |     0.0000 |
| C(founding_cohort)[T.2010-14]       |        0.0000 |    0.0000 |  0.2244 |    0.8225 |    -0.0000 |     0.0000 |
| C(founding_cohort)[T.2015-18]       |        0.0000 |    0.0000 |  1.1879 |    0.2349 |    -0.0000 |     0.0000 |
| C(founding_cohort)[T.2019-20]       |        0.0000 |    0.0000 |  1.9145 |    0.0556 |    -0.0000 |     0.0000 |
| C(founding_cohort)[T.2021]          |        0.0000 |    0.0000 |  5.0499 |    0.0000 |     0.0000 |     0.0000 |
| z_vagueness                         |       -0.0000 |    0.0000 | -1.2597 |    0.2078 |    -0.0000 |     0.0000 |
| z_employees_log                     |        0.0000 |    0.0000 |  7.3989 |    0.0000 |     0.0000 |     0.0000 |

**Key Finding**:
- Vagueness coefficient: -0.0000 (95% CI: [-0.0000, 0.0000])
- p-value: 0.2078
- Interpretation: No significant effect of vagueness on Series A funding
\end{verbatim}

\hypertarget{h2-growth-logistic-regression-with-architecture-moderator}{%
\subsection{H2: Growth (Logistic Regression with Architecture
Moderator)}\label{h2-growth-logistic-regression-with-architecture-moderator}}

\hypertarget{model-specification-1}{%
\subsubsection{Model Specification}\label{model-specification-1}}

\textbf{Research Question}: Does vagueness predict Series B+
progression, and is this moderated by integration cost (architecture)?

\[
P(\text{growth} = 1) = \text{logit}^{-1}(\beta_0 + \beta_1 \cdot z_{\text{vagueness}} + \beta_2 \cdot \text{is\_hardware} + \beta_3 \cdot (z_{\text{vagueness}} \times \text{is\_hardware}) + \text{controls})
\]

\textbf{Where}: - \(\beta_1\): Main effect of vagueness (in
software/modular startups) - \(\beta_3\): Interaction effect
(attenuation/amplification in hardware/integrated startups) -
\textbf{Expected}: \(\beta_3 < 0\) (negative interaction: vagueness
helps more in software)

\hypertarget{results-1}{%
\subsubsection{Results}\label{results-1}}

\hypertarget{h2-results}{}
\begin{verbatim}
#### Coefficient Table

| variable                      |   coefficient |   std_err |        z |   p_value |   ci_lower |   ci_upper |
|:------------------------------|--------------:|----------:|---------:|----------:|-----------:|-----------:|
| Intercept                     |       -1.0903 |    0.0511 | -21.3423 |    0.0000 |    -1.1904 |    -0.9902 |
| C(founding_cohort)[T.2010-14] |       -0.2300 |    0.0581 |  -3.9564 |    0.0001 |    -0.3440 |    -0.1161 |
| C(founding_cohort)[T.2015-18] |       -0.3471 |    0.0534 |  -6.5022 |    0.0000 |    -0.4518 |    -0.2425 |
| C(founding_cohort)[T.2019-20] |       -2.5792 |    0.0902 | -28.6031 |    0.0000 |    -2.7559 |    -2.4025 |
| C(founding_cohort)[T.2021]    |       -2.4034 |    0.2281 | -10.5375 |    0.0000 |    -2.8505 |    -1.9564 |
| z_vagueness                   |       -0.0363 |    0.0193 |  -1.8763 |    0.0606 |    -0.0742 |     0.0016 |
| is_hardware                   |        0.3236 |    0.0424 |   7.6286 |    0.0000 |     0.2405 |     0.4067 |
| z_vagueness:is_hardware       |        0.0925 |    0.0496 |   1.8645 |    0.0622 |    -0.0047 |     0.1897 |
| z_employees_log               |        0.8722 |    0.0199 |  43.9098 |    0.0000 |     0.8333 |     0.9112 |

#### Model Fit Metrics

|       nobs |   prsquared |        aic |        bic |         llf | converged   |    auc |   brier |   logloss |
|-----------:|------------:|-----------:|-----------:|------------:|:------------|-------:|--------:|----------:|
| 40148.0000 |      0.1124 | 33054.2263 | 33131.6292 | -16518.1131 | True        | 0.7394 |  0.1309 |    0.4114 |

#### Average Marginal Effects

| effect          |    value |   std_err |
|:----------------|---------:|----------:|
| AME_z_vagueness | nan      |  nan      |
| slope_level_0   |  -0.0363 |    0.0193 |
| slope_level_1   |   0.0562 |  nan      |

**Interaction Effect**:
- Coefficient: 0.0925 (95% CI: [-0.0047, 0.1897])
- p-value: 0.0622
- Interpretation: Positive interaction (unexpected) - vagueness benefits hardware more
\end{verbatim}

\hypertarget{interaction-plot}{%
\subsubsection{Interaction Plot}\label{interaction-plot}}

\begin{figure}[H]

{\centering \includegraphics[width=0.9\textwidth,height=\textheight]{outputs/bakeoff/h2_interaction_is_hardware.png}

}

\caption{H2 Interaction: Growth × Architecture}

\end{figure}

\textbf{Figure Interpretation}: The interaction plot shows predicted
probabilities of Series B+ progression as a function of vagueness for
hardware vs.~software startups. \textbf{Diverging slopes} indicate
strong moderation by integration cost, while parallel lines suggest weak
interaction.

\textbf{Visual Assessment}:

\hypertarget{visual-assessment}{}
\begin{verbatim}
✗ **Weak/no interaction**: Slopes appear parallel or overlapping
\end{verbatim}

\begin{center}\rule{0.5\linewidth}{0.5pt}\end{center}

\hypertarget{h3h4-architecture-moderator-for-early-funding-growth}{%
\section{H3/H4: Architecture Moderator for Early Funding \&
Growth}\label{h3h4-architecture-moderator-for-early-funding-growth}}

\hypertarget{conceptual-framework}{%
\subsection{Conceptual Framework}\label{conceptual-framework}}

While H2 tests the architecture moderator for growth outcomes, H3 and H4
would extend this moderator to \textbf{both} dependent variables:

\hypertarget{h3-early-funding-architecture-ols}{%
\subsubsection{H3: Early Funding × Architecture
(OLS)}\label{h3-early-funding-architecture-ols}}

\textbf{Research Question}: Does integration cost moderate the vagueness
→ Series A funding relationship?

\textbf{Model Specification}: \[
\text{early\_funding\_musd} \sim z_{\text{vagueness}} * \text{is\_hardware} + \text{z\_employees\_log} + C(\text{founding\_cohort}) + C(\text{sector\_fe})
\]

\textbf{Expected Pattern}: - \textbf{Main effect} (\(\beta_1\)):
Positive for software (vagueness increases funding) -
\textbf{Interaction} (\(\beta_3\)): Negative (attenuation for hardware -
investors demand clarity due to capital intensity)

\textbf{Interpretation}: Hardware startups may face \textbf{earlier
scrutiny} on technical feasibility and market application, making vague
positioning costly even at Series A.

\hypertarget{h4-growth-architecture-logistic-regression}{%
\subsubsection{H4: Growth × Architecture (Logistic
Regression)}\label{h4-growth-architecture-logistic-regression}}

\textbf{Research Question}: Does integration cost moderate the vagueness
→ Series B+ progression relationship?

\textbf{Model Specification}: \[
P(\text{growth} = 1) = \text{logit}^{-1}(\beta_0 + \beta_1 \cdot z_{\text{vagueness}} + \beta_2 \cdot \text{is\_hardware} + \beta_3 \cdot (z_{\text{vagueness}} \times \text{is\_hardware}) + \text{controls})
\]

\textbf{Expected Pattern}: - \textbf{Main effect} (\(\beta_1\)):
Negative for software (vagueness hurts growth) - \textbf{Interaction}
(\(\beta_3\)): Negative (amplification for hardware - compounding risks)

\textbf{Interpretation}: By Series B+, hardware startups with unclear
positioning face \textbf{double jeopardy}: product-market fit
uncertainty PLUS technical execution risk. Software can tolerate more
ambiguity during scaling.

\hypertarget{implementation-status}{%
\subsection{Implementation Status}\label{implementation-status}}

\hypertarget{h3-h4-status}{}
\begin{verbatim}

**H3/H4 with is_hardware moderator**:

📋 **Status**: Conceptually defined, implementation pending

**Next Steps**:
1. Extend `test_h3_early_funding_interaction()` in `models.py` to accept `moderator` parameter
2. Extend `test_h4_growth_interaction()` in `models.py` to accept `moderator` parameter
3. Update `fig_founder_interactions()` in `plots.py` to `fig_moderator_interactions()` (generic)
4. Re-run pipeline with `moderator='is_hardware'` flag

**Expected Timeline**: Can be implemented within 1-2 days once data access is restored.

**Theoretical Priority**:
- If **H2 (architecture) shows strong interaction**, H3/H4 (architecture) become high priority for within-moderator consistency check
- If **H2 (architecture) shows weak interaction**, may prioritize alternative moderators (e.g., founder credibility)
\end{verbatim}

\begin{center}\rule{0.5\linewidth}{0.5pt}\end{center}

\hypertarget{next-steps}{%
\section{Next Steps}\label{next-steps}}

\hypertarget{immediate-priorities}{%
\subsection{Immediate Priorities}\label{immediate-priorities}}

\hypertarget{priorities}{}
\begin{verbatim}
| Priority   | Task                                                    | Rationale                                                  |
|:-----------|:--------------------------------------------------------|:-----------------------------------------------------------|
| High       | Implement H3/H4 with is_hardware moderator              | Complete within-moderator consistency check across DVs     |
| High       | Add founding_year filter to DV construction             | Ensure cohort homogeneity (address Section 4.2 limitation) |
| High       | Generate architecture interaction plots for H3/H4       | Visual assessment of interaction strength                  |
| Medium     | Compare architecture vs. founder credibility moderators | Moderator bake-off for final paper decision                |
\end{verbatim}

\hypertarget{extended-validation}{%
\subsection{Extended Validation}\label{extended-validation}}

\begin{enumerate}
\def\labelenumi{\arabic{enumi}.}
\tightlist
\item
  \textbf{Longer follow-up window}: Re-estimate with 36-48 month
  observation if 2024-2025 data available
\item
  \textbf{External replication}: Validate findings using Crunchbase or
  CB Insights data
\item
  \textbf{Mechanism testing}: Collect qualitative data on investor
  decision-making
\item
  \textbf{Alternative operationalizations}: Test continuous integration
  cost measures (e.g., R\&D intensity, capital expenditures)
\end{enumerate}

\hypertarget{deliverables}{%
\subsection{Deliverables}\label{deliverables}}

\hypertarget{deliverables}{}
\begin{verbatim}
| File                              | Description                                    | Status                   |
|:----------------------------------|:-----------------------------------------------|:-------------------------|
| h1_coefficients.csv               | H1 parameter estimates (OLS)                   | ✓ Generated              |
| h2_model_architecture.csv         | H2 architecture moderator coefficients (Logit) | ✓ Generated              |
| h2_model_architecture_metrics.csv | H2 architecture model fit statistics           | ✓ Generated              |
| h2_interaction_is_hardware.png    | H2 architecture interaction visualization      | ✓ Generated              |
| h3_architecture_coefficients.csv  | H3 architecture interaction estimates (OLS)    | ○ Pending implementation |
| h4_architecture_coefficients.csv  | H4 architecture interaction estimates (Logit)  | ○ Pending implementation |
| Figure_2a_H3_Architecture.png     | H3 architecture interaction visualization      | ○ Pending implementation |
| Figure_2b_H4_Architecture.png     | H4 architecture interaction visualization      | ○ Pending implementation |
\end{verbatim}

\begin{center}\rule{0.5\linewidth}{0.5pt}\end{center}

\hypertarget{references}{%
\section{References}\label{references}}

Baldwin, C. Y., \& Clark, K. B. (2000). \emph{Design rules: The power of
modularity}. MIT Press.

El-Zayaty, A., Hsu, D. H., \& Roberts, E. B. (2025). Linguistic
ambiguity and early-stage venture capital funding. \emph{Strategic
Entrepreneurship Journal}, \emph{19}(1), 45-72.

Gompers, P., Gornall, W., Kaplan, S. N., \& Strebulaev, I. A. (2020).
How do venture capitalists make decisions? \emph{Journal of Financial
Economics}, \emph{135}(1), 169-190.

Granqvist, N., Grodal, S., \& Woolley, J. L. (2013). Hedging your bets:
Explaining executives' market labeling strategies in nanotechnology.
\emph{Organization Science}, \emph{24}(2), 395-413.

Henderson, R. M., \& Clark, K. B. (1990). Architectural innovation: The
reconfiguration of existing product technologies and the failure of
established firms. \emph{Administrative Science Quarterly},
\emph{35}(1), 9-30.

Huang, L., Pearce, J. L., \& Knight, A. P. (2014). Resources and
relationships in entrepreneurship: An exchange theory of the development
and effects of the entrepreneur-investor relationship. \emph{Academy of
Management Review}, \emph{40}(1), 73-95.

Loughran, T., \& McDonald, B. (2016). Textual analysis in accounting and
finance: A survey. \emph{Journal of Accounting Research}, \emph{54}(4),
1187-1230.

Navis, C., \& Glynn, M. A. (2010). How new market categories emerge:
Temporal dynamics of legitimacy, identity, and entrepreneurship in
satellite radio, 1990-2005. \emph{Administrative Science Quarterly},
\emph{55}(3), 439-471.

Navis, C., Fisher, G., Raffaelli, R., Glynn, M. A., \& Watkiss, L.
(2023). The semantics of categories: A critical review and paths forward
for research on category meaning. \emph{Strategic Organization},
\emph{21}(1), 194-224.

Negro, G., \& Leung, M. D. (2013). ``Actual'' and perceptual effects of
category spanning. \emph{Organization Science}, \emph{24}(3), 684-696.

Pan, Y., Siegel, S., \& Wang, T. Y. (2018). The cultural origin of CEOs'
attitudes toward uncertainty: Evidence from corporate acquisitions.
\emph{Review of Financial Studies}, \emph{31}(7), 2977-3030.

Wry, T., Lounsbury, M., \& Jennings, P. D. (2014). Hybrid vigor:
Securing venture capital by spanning categories in nanotechnology.
\emph{Academy of Management Journal}, \emph{57}(5), 1309-1333.

Zuckerman, E. W. (1999). The categorical imperative: Securities analysts
and the illegitimacy discount. \emph{American Journal of Sociology},
\emph{104}(5), 1398-1438.

\begin{center}\rule{0.5\linewidth}{0.5pt}\end{center}

\hypertarget{appendix-model-specification-table}{%
\section{Appendix: Model Specification
Table}\label{appendix-model-specification-table}}

\hypertarget{model-spec-table}{}
\begin{verbatim}
| Hypothesis   | Dependent Variable                  | Model Type   | Key Predictors                       | Sample Filter           | Status        |
|:-------------|:------------------------------------|:-------------|:-------------------------------------|:------------------------|:--------------|
| H1           | early_funding_musd (continuous, $M) | OLS          | z_vagueness + controls               | Early Stage VC only     | ✓ Implemented |
| H2           | growth (binary: B+ progression)     | Logistic     | z_vagueness * is_hardware + controls | Series A at-risk cohort | ✓ Implemented |
| H3*          | early_funding_musd (continuous, $M) | OLS          | z_vagueness * is_hardware + controls | Early Stage VC only     | ○ Pending     |
| H4*          | growth (binary: B+ progression)     | Logistic     | z_vagueness * is_hardware + controls | Series A at-risk cohort | ○ Pending     |

**Controls (all models)**:
- z_employees_log: Firm size (log-transformed, standardized)
- C(founding_cohort): Founding year fixed effects
- C(sector_fe): Industry fixed effects (H1/H3 only)

**Estimation notes**:
- OLS models use robust standard errors (HC3)
- Logistic models use maximum likelihood with L2 regularization (α=0.01) as fallback
- All continuous predictors are z-scored (mean=0, SD=1)

* H3/H4 with is_hardware moderator are conceptually defined and pending implementation
\end{verbatim}

\begin{center}\rule{0.5\linewidth}{0.5pt}\end{center}

\textbf{End of Report}



\end{document}
